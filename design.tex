\documentclass[11pt,a4paper]{article}
\usepackage[margin=1in]{geometry}
\usepackage{hyperref}
\usepackage{enumitem}
\usepackage{graphicx}
\usepackage{listings}
\usepackage{xcolor}
\usepackage{caption}
\usepackage{amsmath}
\usepackage{amssymb}
\usepackage{array}

\definecolor{codebg}{RGB}{248,248,248}
\definecolor{codeborder}{RGB}{220,220,220}
\lstdefinestyle{code}{
  backgroundcolor=\color{codebg},
  basicstyle=\ttfamily\small,
  frame=single,
  rulecolor=\color{codeborder},
  breaklines=true,
  columns=fullflexible,
  keepspaces=true,
  showstringspaces=false,
}

\title{MyTerminal Design Document}
\author{MyTerminal Team}
\date{\today}

\begin{document}
\maketitle

\begin{abstract}
MyTerminal is a lightweight graphical terminal application implemented with X11/Xlib, optionally accelerated by Pango/Cairo for robust UTF-8 rendering. It provides a shell-like environment supporting pipelines, redirections, history, tabs, background jobs, and a custom \texttt{multiWatch} command that executes multiple commands in parallel and streams their outputs with timestamps. This document details the architecture, components, execution model, data flows, and key design decisions.
\end{abstract}

\tableofcontents

\section{Introduction}

\subsection{Overview}
MyTerminal combines a minimal graphical terminal UI with a shell runner:
\begin{itemize}[leftmargin=*]
  \item X11-based windowing, custom rendering and input handling
  \item Command execution via fork/exec with pipes or PTY when interactive
  \item Persistent command history, search, and basic autocompletion
  \item Tabs with independent job state
  \item Background job support
  \item \textbf{multiWatch}: run N commands in parallel per period; stream outputs with UNIX timestamps and per-command headers; clean up on interrupt/exit
\end{itemize}

\subsection{Repository Structure}
\begin{itemize}[leftmargin=*]
  \item \texttt{include/}
    \begin{itemize}
      \item \texttt{gui/TerminalWindow.hpp}: Main app class and public methods
      \item \texttt{gui/Tab.hpp}: Per-tab UI and execution state
      \item \texttt{core/History.hpp}: History model with persistence and search
    \end{itemize}
  \item \texttt{src/}
    \begin{itemize}
      \item \texttt{gui/TerminalWindow.cpp}: X11 window, event loop, drawing, input, scheduling
      \item \texttt{gui/Tab.cpp}: Text rendering helpers and small utilities
      \item \texttt{core/CommandExecutor.cpp}: Command parsing, built-ins, pipelines, processes, multiWatch, jobs
      \item \texttt{core/History.cpp}: History implementation
      \item \texttt{app/main.cpp}: Program entry (includes exit-time sweeper)
    \end{itemize}
  \item \texttt{Makefile}, \texttt{Makefile.nopango}, \texttt{CMakeLists.txt}: Build scripts (Makefile uses Pango/Cairo; Makefile.nopango disables Pango/Cairo; CMake toggles via \texttt{USE\_PANGO\_CAIRO}=ON|OFF)
  \item \texttt{temp/}: Runtime temporary FIFOs used by \texttt{multiWatch}
\end{itemize}

\subsection{Requirements}
\subsubsection{Functional}
\begin{itemize}[leftmargin=*]
  \item Execute arbitrary commands, including pipelines (\verb!cmd1 | cmd2!), with redirection (\verb!>!, \verb!<!, \verb!2>!).
  \item Maintain persistent command history (load at startup, append on every command) and provide inline search (Ctrl+R).
  \item Provide basic autocomplete for built-in commands, external executables, and file/directory paths via Tab key.
  \item Support multiline unicode input, handling characters from various languages and preserving encoding in display.
  \item Provide line navigation shortcuts: Ctrl+A to move cursor to the beginning of the line, Ctrl+E to the end.
  \item Support multi-line input via unmatched quotes and line-continuation (trailing \verb!\!); treat it as a single logical command.
  \item Support multiple commands in one submission (newline-separated or semicolon-separated outside quotes) and execute them sequentially.
  \item Provide background detaching via Ctrl+Z; continue to drain and print background output. The '\&' operator is not supported.
  \item Provide a custom \texttt{multiWatch} command that:
    \begin{itemize}
      \item Executes N commands in parallel for each period.
      \item Creates a hidden temporary FIFO per child using the child's real PID: \verb!temp/.temp.<PID>.txt!.
      \item Reads all child outputs via \texttt{poll()} over those FIFO descriptors and streams output as it arrives.
      \item Prints per-command headers with UNIX timestamp and separator framing.
      \item Cleans up on Ctrl+C: terminate cycle children, unlink all FIFOs, exit worker; restore scrollback in UI.
      \item Removes lingering temp files when the shell closes (\texttt{atexit} sweep) and also sweeps on worker startup.
    \end{itemize}
  \item Render ANSI-colored output; optionally use Pango/Cairo for robust UTF-8 shaping.
  \item Provide tabs and basic UI affordances (scrollbar, prompt, separators between pasted commands).
\end{itemize}

\subsubsection{Non-Functional}
\begin{itemize}[leftmargin=*]
  \item Linux/X11 environment; C++17.
  \item Single-process GUI (no threads) with nonblocking I/O to keep UI responsive.
  \item Clean failure modes: if exec fails, report errors; avoid orphaned FIFOs; sweep temp directory on exit/start.
  \item Reasonable performance for typical command output and up to dozens of \texttt{multiWatch} commands.
\end{itemize}

\section{Core Architecture and Execution Model}

\subsection{High-Level Components}
\begin{description}
  \item[TerminalWindow] X11 window, event loop, input handling, command scheduling, rendering pipeline.
  \item[Tab] Per-tab state: buffers, job descriptors, queues, continuation state, multiWatch snapshot.
  \item[Command Execution] Parser + launcher for built-ins, pipelines, PTY selection, and process I/O wiring.
  \item[multiWatch Worker] A forked child that orchestrates per-period parallel children and streams via FIFOs.
  \item[History] Persistent store for commands; search and best-match suggestions.
\end{description}

\subsection{Event Loop and Scheduling}
\begin{itemize}[leftmargin=*]
  \item The X11 loop handles input events (KeyPress, Mouse), periodic repaint ticks, and file-descriptor polling for child output.
  \item \texttt{runNextCommand()} dequeues and dispatches commands sequentially; built-ins return quickly and chain to next; external commands advance when reaped.
  \item Between commands from a pasted batch, a visible separator line is printed to avoid output confusion.
\end{itemize}

\subsection{Data Flow (typical external command)}
\begin{enumerate}[leftmargin=*]
  \item User submits text; it is split into logical commands while respecting quotes and semicolons.
  \item A command is parsed into pipeline stages with redirections.
  \item The process tree is forked; pipes/PTY are wired.
  \item Parent monitors nonblocking read ends; chunks are appended to \texttt{scrollback} after ANSI processing.
  \item On EOF and successful \texttt{waitpid()}, scheduling advances.
\end{enumerate}

\subsection{TerminalWindow}
Primary UI controller (see \texttt{TerminalWindow.hpp}):
\begin{itemize}[leftmargin=*]
  \item X11 setup, event loop, drawing: tab bar, text area, scrollbar
  \item Input handling: UTF-8 IME (XIM/XIC), line editing, history search (Ctrl+R), autocomplete
  \item Command submission and scheduling across \texttt{Tab::pendingCmds}
  \item Process I/O pump: nonblocking read from child stdout/stderr; background drains
  \item Prompting and transcript building with simple ANSI parsing and coloring
  \item Optional Pango/Cairo rendering for multilingual text
\end{itemize}

\subsection{Tab}
Per-tab state container (\texttt{Tab.hpp}):
\begin{itemize}[leftmargin=*]
  \item Text buffers: \texttt{scrollback}, \texttt{input}
  \item Cursor and scrolling metrics
  \item Foreground job: \texttt{childPid}, \texttt{childPgid}, and FDs
  \item Background jobs: list of \texttt{BackgroundJob}
  \item Continuation state for multi-line commands (quotes/backslashes)
  \item Queue: \texttt{pendingCmds} for sequential execution
  \item multiWatch session flag and saved scrollback snapshot
\end{itemize}

\subsection{Command Submission and Scheduling}
\begin{enumerate}[leftmargin=*]
\item The user enters a line; MyTerminal echoes the prompt and line.
\item The input may contain:
  \begin{itemize}
    \item Multiple commands separated by newlines or semicolons (outside quotes)
    \item Continuation lines for unmatched quotes or line-continuation backslashes
  \end{itemize}
\item Parsed commands are enqueued into \texttt{Tab::pendingCmds}.
\item \texttt{runNextCommand()} dequeues and dispatches one at a time:
  \begin{itemize}
    \item Built-ins execute synchronously and now call \texttt{runNextCommand()} when done.
    \item External programs launch via fork/exec. Completion is detected through nonblocking I/O pump and \texttt{waitpid()} (WNOHANG); then \texttt{runNextCommand()} continues.
  \end{itemize}
\item For clarity with multi-command pastes, a visual separator line is printed between commands.
\end{enumerate}


\section{Feature: Command Parsing and Multiline Input}

\subsection{Splitting Lines and Commands}
\paragraph{Split by newline respecting quotes.} We walk bytes and toggle flags \texttt{inS/inD} for single/double quotes. Newlines outside quotes delimit commands. Whitespace-only fragments are discarded.
\begin{lstlisting}[style=code]
inS=false, inD=false, cur=""
for c in input:
  if c=='"' && !inS: inD=!inD; cur+=c; continue
  if c=='\'' && !inD: inS=!inS; cur+=c; continue
  if c=='\n' && !inS && !inD: push(cur if non-empty non-whitespace); cur=""; continue
  cur+=c
push(cur if non-empty non-whitespace)
\end{lstlisting}

\paragraph{Split by semicolon respecting quotes.} Applied per logical line for sequences like \verb!cmd1; cmd2!.

\subsection{Multi-line Input and Continuation}
\begin{itemize}[leftmargin=*]
  \item Support multi-line input via unmatched quotes and line-continuation (trailing \verb!\!); treat it as a single logical command.
  \item Continuations show \verb!> ! prefix; transcript preserves exact typed content (excluding continuation backslashes).
  \item Continuation with unmatched quotes and \texttt{\textbackslash} line-joins retains logical command structure.
\end{itemize}


\section{Feature: Persistent History and Line Editing}

\subsection{History Implementation}
(\texttt{History.hpp/.cpp}) Persistent ring with:
\begin{itemize}[leftmargin=*]
  \item Append, clear, save/load to \texttt{\textasciitilde/.myterm\_history}: History is stored in a plain text file in the user's home directory, with each command on a new line. The file is loaded at startup into a vector for quick access. Commands are appended after successful execution, and the file is saved periodically or on exit. The \texttt{clear} subcommand resets the in-memory history and truncates the file.
  \item Search with exact and substring strategies: Exact search matches commands that start with the query. Substring search uses longest common subsequence (LCS) to find commands containing the query characters in order, prioritizing higher LCS lengths.
  \item Best-match suggestions for autocomplete: When autocompleting, history provides suggestions based on prefix matches.
\end{itemize}

\subsection{Line Editing and Search}
\begin{itemize}[leftmargin=*]
  \item Line editing supports basic input; Ctrl+A (start of line) and Ctrl+E (end of line): The input line is a string buffer with a cursor position. Characters are inserted at the cursor, and Ctrl+A moves the cursor to index 0, Ctrl+E to the end of the string.
  \item Ctrl+R searches backward in command history: Initiates reverse search mode, where typing narrows down matches from history. Up/down arrows cycle through matches, and Enter selects one to fill the input line.
  \item Backspace/Delete edit text: Backspace removes the character before the cursor and shifts the rest left; Delete removes the character at the cursor.
\end{itemize}

\section{Feature: Autocomplete Functionality}

\subsection{Design}
\begin{itemize}[leftmargin=*]
  \item When the user presses the Tab key during input, the shell attempts to complete the current word.
  \item If the word matches the prefix of a built-in command, external command, or file/directory in the current working directory, possible completions are suggested or inserted.
  \item Directory-prefix-aware replacements are supported: typing a partial directory or file name and pressing Tab will complete or list matches.
  \item For file name completion:
    \begin{itemize}
      \item If one file matches the prefix, complete the input with the full file name.
      \item If multiple files match, complete to the longest common prefix if it extends the current input.
      \item If multiple matches remain after prefix completion, display numbered options (e.g., 1. file1.txt 2. file2.txt) and prompt the user to select by number.
    \end{itemize}
\end{itemize}

\subsection{Implementation}
\begin{itemize}[leftmargin=*]
  \item The input handler scans the current input buffer for the word under the cursor.
  \item Built-in commands are matched from a static list; external commands are matched from the system PATH.
  \item File and directory matches are found using globbing in the current working directory.
  \item The transcript is updated with suggestions if multiple matches are found; otherwise, the input buffer is updated with the completed word.
\end{itemize}

\subsection{Limitations}
\begin{itemize}[leftmargin=*]
  \item Autocomplete does not support advanced shell grammar (e.g., variable expansion, command substitution).
  \item Only single-word completion is supported; multi-stage pipeline or quoted arguments may not autocomplete as expected.
  \item No fuzzy matching; only prefix matches are considered.
\end{itemize}

\subsection{Usage}
\begin{itemize}[leftmargin=*]
  \item Press Tab while typing a command or file path to trigger autocomplete.
  \item If multiple completions are possible, they will be listed in the transcript area.
  \item Example: typing \texttt{ec} and pressing Tab will complete to \texttt{echo}.
  \item Example: typing \texttt{test\_fi} and pressing Tab will complete to \texttt{test\_file.txt} if present in the directory.
  \item Example (file completion): In a directory with files "abc.txt", "def.txt", "abcd.txt":
    \begin{itemize}
      \item Typing \texttt{./myprog de} and pressing Tab completes to \texttt{./myprog def.txt}.
      \item Typing \texttt{./myprog abc} and pressing Tab displays "1. abc.txt 2. abcd.txt" and waits for user input (e.g., pressing 1 completes to \texttt{./myprog abc.txt}).
    \end{itemize}
\end{itemize}

\section{Feature: Pipelines and I/O Redirection}

\subsection{Parsing and Wiring}
\begin{enumerate}[leftmargin=*]
  \item Parse stages, and within each stage, extract redirections for stdin/stdout/stderr.
  \item Create N-1 pipes for an N-stage pipeline.
  \item For single-stage with no redirection, attempt PTY for interactive behavior; else use pipes.
  \item Child i: dup2 appropriate fds (stdin from prev pipe or file; stdout to next pipe or parent outPipe; stderr to errPipe unless redirected).
  \item Close unused fds in children and parent; parent sets nonblocking on read ends.
\end{enumerate}

\subsection{PTY Usage}
\begin{itemize}[leftmargin=*]
  \item PTY mode is used for single-stage interactive commands (TTY-friendly behavior); otherwise, pipes are used.
\end{itemize}

\section{Feature: Background Jobs and Signal Handling}

\subsection{Design}
\begin{itemize}[leftmargin=*]
  \item Background jobs are processes detached from the foreground, allowing continued execution while the shell accepts new commands.
  \item Ctrl+Z detaches the current foreground job, moving it to background; output is asynchronously drained and displayed.
  \item Jobs are tracked by PID/PGID and file descriptors for output monitoring.
  \item Built-in commands \texttt{bgpids} list active background jobs, and \texttt{killprocess} sends signals to terminate them.
  \item Signal handling: Ctrl+C interrupts foreground process groups; multiWatch sets its own PGID for targeted interruption.
\end{itemize}

\subsection{Implementation}
\begin{itemize}[leftmargin=*]
  \item Ctrl+Z handler: Sends SIGTSTP to foreground PGID, adds job to \texttt{backgroundJobs} vector with PID/PGID/FDs.
  \item \texttt{drainBackgroundJobs()}: Nonblocking read from background FDs, appends output to scrollback.
  \item \texttt{bgpids}: Iterates \texttt{backgroundJobs}, prints PID and command name.
  \item \texttt{killprocess}: Parses PID and signal (default SIGTERM, -9 for SIGKILL), sends via \texttt{kill}.
  \item Process groups: Foreground jobs set PGID; Ctrl+C uses \texttt{killpg} to signal group.
\end{itemize}

\subsection{Limitations}
\begin{itemize}[leftmargin=*]
  \item No job control commands like \texttt{fg} or \texttt{jobs}.
  \item No '\&' operator for backgrounding at command end.
\end{itemize}

\subsection{Usage}
\begin{itemize}[leftmargin=*]
  \item Ctrl+Z: Detach running command to background.
  \item \texttt{bgpids}: List background jobs with PIDs and commands.
  \item \texttt{killprocess PID}: Terminate a background job (use -9 for force kill).
\end{itemize}

\subsection{Background Detaching (Ctrl+Z)}
Ctrl+Z detaches the current foreground job; output is drained and appended to the transcript. The \texttt{\&} operator is not supported. Jobs are tracked with PID/PGID and FDs.

\subsection{Signal and Process Group Management}
\begin{itemize}[leftmargin=*]
  \item Foreground jobs form a process group; Ctrl+C sends SIGINT via \texttt{killpg}.
  \item multiWatch worker sets its own PGID so Ctrl+C targets the whole watch job.
\end{itemize}

\section{Feature: Custom Command \texttt{multiWatch}}

\subsection{Goals}
\begin{itemize}[leftmargin=*]
  \item Run multiple commands \emph{in parallel} each period
  \item Stream outputs intermixed as data is available, not after completion
  \item Tag each stream with command and UNIX timestamp
  \item Clean up cleanly on Ctrl+C and when the shell exits
\end{itemize}

\subsection{Overview}
\begin{enumerate}[leftmargin=*]
\item The shell forks a \textbf{multiWatch worker} process.
\item Each period, the worker:
  \begin{enumerate}
    \item Forks N children; each child writes to a FIFO at \texttt{temp/.temp.\textless PID\textgreater.txt} (one per command).
    \item The worker opens nonblocking read-ends for all FIFOs and uses \texttt{poll()} to multiplex.
    \item As data arrives, it prints:
      \begin{itemize}
        \item Header: \texttt{"cmd" , current\_time: <unix\_timestamp> :}
        \item Separator line: 56 dashes (as configured)
        \item Raw command output (may be empty)
        \item Trailing separator line
      \end{itemize}
    \item On EOF/HUP for a FIFO, the worker closes and unlinks that FIFO.
    \item After all commands finish, the worker sleeps for the configured interval and repeats.
  \end{enumerate}
\item Ctrl+C kills the worker's process group; the worker's signal handler then:
  \begin{itemize}
    \item Kills all child PIDs for the current cycle
    \item Unlinks any FIFOs it knows about
    \item Exits
  \end{itemize}
\item The parent (GUI) reads the worker's stdout and paints the transcript; upon worker exit, it restores the scrollback that was present before multiWatch started.
\end{enumerate}

\subsection{multiWatch Worker Loop Implementation}
\paragraph{Rationale for FIFOs over regular files.} Regular files are always readable by \texttt{poll()}, making readiness semantics unhelpful. FIFOs support readiness and EOF/HUP, enabling true streaming.

\paragraph{Per period algorithm.}
\begin{lstlisting}[style=code]
// At worker startup: sweep temp/.temp.*.txt (orphan cleanup)
loop forever:
  clear mw_pids, mw_tempfiles
  mkdir("temp", 0755)
  // Spawn N children
  for i in 0..N-1:
    p = fork()
    if child:
      tf = "temp/.temp." + getpid() + ".txt"
      // Nonblocking open loop until reader is up
      while ( (wfd=open(tf, O_WRONLY|O_NONBLOCK))<0 ) sleep 10ms
      dup2(wfd, 1); dup2(wfd, 2); close(wfd)
      execlp("sh","sh","-c",cmd[i],NULL)
      _exit(127)
    else:
      tf = "temp/.temp." + p + ".txt"; unlink(tf); mkfifo(tf,0644)
      rfd = retry open(tf, O_RDONLY|O_NONBLOCK)
      pfds.push({fd:rfd, events:POLLIN|POLLHUP|POLLERR}); map pfds->cmdIndex

  // Stream with poll
  headerPrinted[j]=false; trailerPrinted[j]=false
  while openCount>0:
    poll(pfds, 200ms)
    for each j with revents:
      if POLLIN: read; on first bytes print header+separator; write chunk
      if read==0 or POLLHUP/ERR: drain; ensure trailing separator; close+unlink fifo; openCount--

  // Reap children and record exit codes (optional); sleep interval seconds
\end{lstlisting}

\subsection{File/Descriptor Strategy}
The design uses \textbf{FIFOs} for each child invocation:
\begin{itemize}[leftmargin=*]
  \item Satisfies requirement to create hidden temp files named with the real PID
  \item Enables true nonblocking multiplexing with \texttt{poll()} over read descriptors
  \item Files are short-lived: unlinked as soon as the corresponding child completes
\end{itemize}
A sweep is performed at worker startup and at shell exit (via \texttt{atexit}) to remove stale entries in \texttt{temp/}.

\subsection{Cleanup Strategy and Interrupts}
\begin{itemize}[leftmargin=*]
  \item \textbf{Per-stream}: On EOF/HUP, close the FIFO and immediately \texttt{unlink()} it.
  \item \textbf{On signals}: SIGINT/TERM/HUP/QUIT handled by the worker: kill cycle PIDs, unlink all \texttt{mw\_tempfiles}, exit.
  \item \textbf{On worker startup}: Sweep \texttt{temp/.temp.*.txt} via glob and unlink.
  \item \textbf{On app exit}: \texttt{atexit} sweep in \texttt{main.cpp} unlinks lingering \texttt{temp/.temp.*.txt}.
\end{itemize}

\subsection{Formatting}
\paragraph{Header/Separator Formatting.}
On first data (or EOF without data) for a given command in the period, print:
\begin{lstlisting}[style=code]
"<cmd>" , current_time: <unix_timestamp> :
----------------------------------------------------
<data if any>
----------------------------------------------------
\end{lstlisting}
The dashed line count is constant in code and can be tuned; a separate, longer dashed line is used between queued commands from pasted input.

Per command per cycle:
\begin{lstlisting}[style=code]
"cmd" , current_time: 1690000000 :
----------------------------------------------------
<Output>
----------------------------------------------------
\end{lstlisting}
Repeat for each command, and then each cycle.

\section{Feature: Rendering, Unicode, and UI/UX}

\subsection{Multiline Unicode Input and Rendering}
The requirement specifies X11/Xlib for rendering, but X11 alone is insufficient for robust handling of multiline Unicode input, particularly when dealing with characters from various languages that require complex shaping and layout. To address this, MyTerminal optionally uses Pango/Cairo for UTF-8 text rendering, which provides proper glyph shaping, bidirectional text support, and accurate positioning for multilingual content.

For example, if the user types a command like \texttt{echo "Hello 世界"}, the Unicode characters (e.g., Chinese characters) are correctly shaped and displayed in the terminal output, preserving encoding and layout. This is achieved by integrating Pango's text layout engine with Cairo's rendering backend, ensuring that multiline inputs and outputs with mixed scripts are rendered faithfully without corruption.

The build system supports this via the \texttt{USE\_PANGO\_CAIRO} flag in CMake, or by using \texttt{Makefile} (which includes Pango) versus \texttt{Makefile.nopango} (which disables it for minimal builds).

\begin{itemize}[leftmargin=*]
  \item Xlib text drawing with ANSI color approximation; optional Pango/Cairo rendering for true UTF-8 shaping
  \item Custom scrollback and scrollbar; 60 Hz redraw loop for smoothness
  \item Prompt coloring (user/host/cwd) with theme-able palette
  \item ANSI parsing tracks ESC/CSI states and applies color/intensity where feasible.
\end{itemize}

\subsection{UI/UX Details}
\begin{itemize}[leftmargin=*]
  \item Prompt format: \verb!user@host:cwd$ !; colored segments for readability.
  \item Between commands in a multi-command submission, a long separator line is printed: \verb!-------------------------------------------------------------!.
\end{itemize}

\section{Supporting Details}

\subsection{Commands and Shortcuts}
\subsubsection{Built-in Commands}
The shell supports the following built-in commands, which are handled internally without forking external processes:

\begin{description}
  \item[multiWatch] \hfill \\
    Syntax: \texttt{multiWatch [interval] ["cmd1", "cmd2", ...]} or \texttt{multiWatch [interval] cmd1 cmd2 ...} \\
    Runs multiple commands in parallel each period, streaming outputs with headers and timestamps. Interval defaults to 5 seconds if omitted. \\
    Example: \texttt{multiWatch 10 ["date", "uptime"]} \\
    Feature: multiWatch (parallel monitoring)

  \item[bgpids] \hfill \\
    Syntax: \texttt{bgpids} \\
    Lists all active background job PIDs with their command names. \\
    Feature: Background Jobs

  \item[killprocess] \hfill \\
    Syntax: \texttt{killprocess [-9] PID [PID ...]} \\
    Sends SIGTERM (or SIGKILL if -9) to the specified PIDs. \\
    Example: \texttt{killprocess 1234} \\
    Feature: Background Jobs (use Ctrl+Z to detach a running foreground job; '&' is not supported)

  \item[echo] \hfill \\
    Syntax: \texttt{echo [args ...]} \\
    Prints arguments to stdout, separated by spaces. \\
    Feature: Command Execution

  \item[history] \hfill \\
    Syntax: \texttt{history} or \texttt{history clear} \\
    Shows command history; \texttt{clear} removes all history. \\
    Feature: History

  \item[cd] \hfill \\
    Syntax: \texttt{cd [directory]} \\
    Changes current working directory; defaults to home if no argument. \\
    Feature: Command Execution

  \item[clear] \hfill \\
    Syntax: \texttt{clear} \\
    Clears the terminal screen. \\
    Feature: UI/UX

  \item[pwd] \hfill \\
    Syntax: \texttt{pwd} \\
    Prints the current working directory. \\
    Feature: Command Execution (implied in prompt)
\end{description}

\subsubsection{Keyboard Shortcuts}
The following keyboard shortcuts are supported for input editing and control:

\begin{itemize}[leftmargin=*]
  \item \textbf{Ctrl+C}: Interrupt the current foreground command (sends SIGINT to process group).
  \item \textbf{Ctrl+Z}: Send the current foreground command to background (detaches job).
  \item \textbf{Ctrl+A}: Move cursor to the beginning of the line.
  \item \textbf{Ctrl+E}: Move cursor to the end of the line.
  \item \textbf{Ctrl+R}: Search backward in command history.
  \item \textbf{Arrow Keys}: Navigate cursor left/right.
  \item \textbf{Backspace/Delete}: Edit text.
\end{itemize}
These shortcuts enhance usability for line editing, history access, and job control.

\subsubsection{Feature-to-Commands Mapping}
For reference, here is a mapping of features to their associated commands/shortcuts:

\begin{itemize}[leftmargin=*]
  \item \textbf{Command Execution (Pipelines, Redirections)}: External commands (e.g., \texttt{ls | grep foo > out.txt}), \texttt{echo}, \texttt{cd}, \texttt{pwd}.
  \item \textbf{Interactive Mode \& PTY}: Automatic for single-stage commands like \texttt{vim}.
  \item \textbf{Background Jobs}: \texttt{bgpids}, \texttt{killprocess}, Ctrl+Z (the '&' operator is not supported).
  \item \textbf{multiWatch}: \texttt{multiWatch}, Ctrl+C (interrupt).
  \item \textbf{Input Handling}: Multi-line with quotes/backslashes, semicolon/newline separation.
  \item \textbf{History}: \texttt{history}.
  \item \textbf{Signal Handling \& Cleanup}: Ctrl+C, Ctrl+Z.
  \item \textbf{Rendering \& Output}: ANSI colors, \texttt{clear}, Ctrl+L.
  \item \textbf{Temp Files \& Resource Management}: Automatic (FIFOs for multiWatch).
  \item \textbf{Build \& Configuration}: \texttt{make}, \texttt{cmake}.
  \item \textbf{UX Enhancements}: Separators, prompts, shortcuts.
\end{itemize}

\subsection{Error Handling and Edge Cases}
\begin{itemize}[leftmargin=*]
  \item \textbf{Command not found}: prints \emph{command not found} with status 127
  \item \textbf{Globbing}: expanded via \texttt{glob(3)}; tokens not matching become literals
  \item \textbf{UTF-8 input}: Accepted via XIM/XIC; unrecognized control bytes are dropped
  \item \textbf{PTY fallback}: If PTY allocation fails, falls back to pipe mode
  \item \textbf{multiWatch FIFO races}: Reader open retries and nonblocking write open in child avoid ENXIO/ENOENT races
\end{itemize}

\subsection{Performance Considerations}
\begin{itemize}[leftmargin=*]
  \item Nonblocking I/O and short poll timeouts keep UI responsive.
  \item \texttt{multiWatch} scales roughly O(N) in number of polled FIFOs; each poll loop is bounded and data is chunked.
  \item PTY path is used for single-stage interactive commands to avoid line-buffering surprises and to support TTY-aware programs.
\end{itemize}

\subsection{Build and Configuration}
Two build routes:
\begin{itemize}[leftmargin=*]
  \item \textbf{Make (with or without Pango/Cairo)}:
\begin{lstlisting}[style=code]
# With Pango/Cairo (default Makefile uses Pango)
make -f Makefile
make -f Makefile clean

# Without Pango/Cairo
make -f Makefile.nopango
make -f Makefile.nopango clean
\end{lstlisting}
  \item \textbf{CMake}:
\begin{lstlisting}[style=code]
cmake -S . -B build -DUSE_PANGO_CAIRO=ON   # or OFF to disable Pango/Cairo
cmake --build build
\end{lstlisting}
\end{itemize}
For CMake builds, toggle Pango/Cairo via the cache option \texttt{-DUSE\_PANGO\_CAIRO=ON|OFF}.

\subsubsection{Runtime Paths and Permissions}
\begin{itemize}[leftmargin=*]
  \item Temp FIFOs live under \texttt{temp/} with filenames \texttt{.temp.<PID>.txt}. Default mode 0644 is adequate for single-user project work; production-grade terminals should place temp under \texttt{/tmp/myterm-<uid>} with directory mode 0700 to avoid cross-user interference.
  \item \texttt{umask} may affect FIFO file modes; code assumes a permissive default.
\end{itemize}

\subsection{Security Considerations}
\begin{itemize}[leftmargin=*]
  \item No shell escaping is performed beyond glob expansion; commands are executed via execvp or \texttt{sh -c} (in multiWatch). Treat input as untrusted.
  \item Temp FIFOs are created under project \texttt{temp/}; production usage should prefer \texttt{/tmp/myterm-<uid>} with \texttt{0700} to avoid cross-user interference.
  \item Signal handling avoids partial cleanup by unlinking known FIFOs and killing child PIDs on exit paths.
\end{itemize}

\subsection{Testing Strategy}
\begin{itemize}[leftmargin=*]
  \item Unit test core string parsing where feasible (history, splitting helpers)
  \item Integration tests:
    \begin{itemize}
      \item Pipes/redirections correctness
      \item PTY execution for interactive programs
      \item multiWatch with short/long commands; verify headers, timestamps, and cleanup
      \item Signal handling on Ctrl+C and shell exit: ensure no \texttt{temp/.temp.*.txt} remains
    \end{itemize}
\end{itemize}

\subsection{Limitations and Future Work}
\begin{itemize}[leftmargin=*]
  \item Full POSIX job control is limited; background output is supported but lacks job control commands like \texttt{fg}/\texttt{jobs}.
  \item No shell scripting language---this is a runner with simple parsing, not a full shell grammar.
  \item multiWatch currently uses \texttt{sh -c}; direct argv variants could avoid shell interpolation.
  \item Optional enhancements:
    \begin{itemize}
      \item \texttt{--keep-temp}: keep FIFOs/files for debugging instead of unlinking
      \item Persistent per-command logs via parent-side \texttt{tee}
      \item Configurable separators and header formats
      \item More robust ANSI parsing and colors
      \item Better prompt customization and themes
    \end{itemize}
\end{itemize}

\section{Appendix: Key APIs}
\subsection{TerminalWindow}
\begin{lstlisting}[style=code]
void run();
void executeLine(const std::string& line);
void executeLineInternal(const std::string& line, bool echoPromptAndCmd);
void spawnProcess(const std::vector<std::string>& argv);
void pumpChildOutput();
void drainBackgroundJobs();
void submitInputLine(Tab& t, bool triggerRedraw = true);
void runNextCommand(Tab& t);
static std::vector<std::string> splitArgs(const std::string& s);
\end{lstlisting}

\subsection{Tab}
\begin{lstlisting}[style=code]
std::deque<std::pair<std::string,bool>> pendingCmds;
pid_t childPid, childPgid;
int outFd, errFd, inFdWrite;
bool watchActive;
std::string savedScrollbackBeforeWatch;
\end{lstlisting}

\subsection{multiWatch (worker inner loop)}
\begin{lstlisting}[style=code]
// For each period:
- Fork command children
- Parent mkfifo temp/.temp.<PID>.txt and opens read end (nonblocking)
- Child opens write end and dup2s stdout/stderr to it
- Parent poll()s across all FIFO read fds:
  - Print header, then separator, then stream data, then trailing separator
- Close and unlink FIFO when done
- Sleep interval seconds and repeat
\end{lstlisting}
\end{document}